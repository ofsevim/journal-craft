\documentclass{scd}
\setlang{EN} % Belge dilini burada EN veya TR olarak belirleyebilirsiniz

% --- Journal-level metadata ---
\journalname{Sosyal Çalışma Dergisi}
\journalurl{https://dergipark.org.tr/tr/pub/scd}
\articletype{Araştırma} % Sistematik Derleme, Araştırma, Derleme
\eissn{2587-1412}
\monthyearTR{Aralık 2025}
\monthyearEN{December 2025}
\volume{9}
\issue{2}

% --- Article-level metadata ---
\titleTR{Engellilik ve Sosyal Hizmet: Sistematik Bir Derleme}
\titleEN{Disability and Social Work: A Systematic Review}

% Authors - Continuous Line Format
% Authors - Continuous Line Format
\authorname{%
Mehmet Gedik\authorsup{1} \href{https://orcid.org/0000-0002-1234-5678}{\SCDORCIDIcon} 
Kaan Sevim\authorsup{2} \href{https://orcid.org/0000-0001-9876-5432}{\SCDORCIDIcon} 
Ayşe Demir\authorsup{3} \href{https://orcid.org/0000-0003-1111-2222}{\SCDORCIDIcon} ve 
Mehmet Kaya\authorsup{4} \href{https://orcid.org/0000-0004-3333-4444}{\SCDORCIDIcon}
}
\initialsauthors{SEVİM ve ark.}

\authoraffiliation{%
  \noindent\makebox[0pt][r]{\authorsup{1}} Uzman Hemşire– İstanbul Aydın Üniversitesi, Lisansüstü Eğitim Enstitüsü, İstanbul, Türkiye \SCDRORIcon{https://ror.org/01y3j3k12}\par\vspace{1pt}
  \noindent\makebox[0pt][r]{\authorsup{2}} İstanbul Üniversitesi-Cerrahpaşa \SCDRORIcon{https://ror.org/037zpmp32}\par\vspace{1pt}
  \noindent\makebox[0pt][r]{\authorsup{3}} Ankara Üniversitesi \SCDRORIcon{https://ror.org/014231649}\par\vspace{1pt}
  \noindent\makebox[0pt][r]{\authorsup{4}} İstanbul Üniversitesi-Cerrahpaşa \SCDRORIcon{https://ror.org/037zpmp32}
}

\correspondingauthor{Seda Eroğlu}{sedaeroglu@uludag.edu.tr}

\submittedTR{12.09.2025}
\acceptedTR{25.12.2025}
\publishedTR{30.12.2025}

\keywordsTR{Engellilik, Sosyal hizmet, Algı, Mesleki müdahale, Toplumsal kabul}
\keywordsEN{Disability, Social Work, Perception, Professional Intervention, Social Acceptance}

\doi{10.70989/scd.1692064}
\ethicsTR{Muğla Sıtkı Koçman Üniversitesi 03.10.2023 tarih 90 karar numaralı etik kurul izni alınmıştır.}
\ethicsEN{Ethics committee approval was obtained from Muğla Sıtkı Koçman Üniversitesi, dated 03.10.2023, decision number 90.}

\abstractTR {Engellilik algısı, sosyal hizmet çalışanlarının mesleki uygulamalarını ve hizmet sunum süreçlerini şekillendiren temel unsurlardan biridir. Bu araştırmanın amacı sosyal hizmet çalışanlarının engellilere yönelik algılarını değerlendirmek ve bu algıların demografik değişkenlere göre farklılık gösterip göstermediğini analiz etmektir. Açıklayıcı tipte hazırlanan bu çalışmada nicel temelli tarama deseni kullanılmıştır. Araştırma 250 sosyal hizmet çalışanı örnekleminde yürütülmüş ve katılımcılara anket veri toplama aracıyla ulaşılmıştır. Araştırmada toplanan verilerin analizinde IBM SPSS 22.00 yazılımından faydalanılmış, demografik özelliklere göre gruplar arasındaki farkları incelemek amacıyla iki grup arasındaki karşılaştırmalar için bağımsız iki örnek t-test, ikiden fazla grubu karşılaştırmak için tek yönlü ANOVA analizleri yapılmıştır. Elde edilen bulgular sosyal hizmet çalışanlarının engellilere yönelik algılarının orta düzey olduğunu ve bu algıların cinsiyet ile yaş değişkenleri açısından istatistiksel olarak anlamlı farklılık gösterdiğini ortaya koymaktadır. Bu farklılıklar sosyal hizmet çalışanlarının engellilere ilişkin algılarının homojen bir yapı sergilemediğini ve engelliliğe yönelik mesleki müdahale, uygulama ve karar alma süreçlerinde bireysel değişkenlerin etkili olabileceğini göstermektedir.}

\abstractEN{Disability perception constitutes a critical component shaping the professional practices and service delivery strategies of social service professionals. This study seeks to explore how social service workers perceive disability and to determine whether these perceptions vary based on demographic factors. Employing a descriptive and quantitative research design, data were collected from 250 social service professionals through a structured questionnaire. The dataset was analysed using IBM SPSS 22.0 software. Independent samples t-tests were conducted to compare binary groups, while one-way ANOVA was applied for comparisons across multiple demographic categories. The results indicate that participants generally hold a moderate level of disability perception, with significant differences observed across gender and age groups. These findings suggest that social service workers' perceptions of disability are not uniform and may be influenced by individual characteristics, which in turn can affect professional interventions and decision-making processes in disability-related services. These findings suggest that social service workers' perceptions of disability are not uniform and may be influenced by individual. }

\citationtext{Eroğlu, S., Yılmaz, A., Demir, A., \& Kaya, M. (2025). Engellilik ve Sosyal Hizmet: Sistematik Bir Derleme. \textit{Sosyal Çalışma Dergisi, 9}(2), 61--74. \href{https://doi.org/10.70989/scd.1692064}{https://doi.org/10.70989/scd.1692064}}

% --- Sayfa Numarası Başlangıcı ---
\startpage{65} % Makalenin başlayacağı sayfa numarası

\begin{document}

\maketitle

\begin{scdbody}

\section{GİRİŞ}
Engellilik, bireyin fiziksel, zihinsel, duyusal veya psikolojik farklılıkları nedeniyle toplumsal yaşama tam ve etkin katılımında karşılaştığı engeller bütünü olarak tanımlanmaktadır. Sosyal hizmet disiplini, engellilik olgusunu biyo-psiko-sosyal bir perspektifle ele alarak, bireylerin güçlendirilmesi ve toplumsal dışlanmanın önlenmesi adına çeşitli müdahale stratejileri geliştirmektedir.

Engellilik, bireyin fiziksel, zihinsel, duyusal veya psikolojik farklılıkları nedeniyle toplumsal yaşama tam ve etkin katılımında karşılaştığı engeller bütünü olarak tanımlanmaktadır. Sosyal hizmet disiplini, engellilik olgusunu biyo-psiko-sosyal bir perspektifle ele alarak, bireylerin güçlendirilmesi ve toplumsal dışlanmanın önlenmesi adına çeşitli müdahale stratejileri geliştirmektedir.

\section{YÖNTEM}
Bu çalışmada sistematik derleme yöntemi kullanılmıştır. Veri toplama sürecinde aşağıdaki akış şeması izlenmiştir.

\section{BULGULAR}
Araştırma bulguları, sosyal hizmet çalışanlarının engellilik algısının cinsiyet, yaş ve mesleki deneyim yılına göre değişkenlik gösterdiğini ortaya koymuştur.

% ===========================================================
% ÖRNEK 1: İKİ SÜTUNLU (NORMAL) TABLO
% Açıklama: Bu tablo iki sütunlu metin akışı içinde kalır.
% Kullanım: Küçük/orta boyutlu tablolar için idealdir.
% ===========================================================
\begin{table}[H]
\centering
\caption{Katılımcıların Demografik Özellikleri}
\begin{threeparttable}
\label{tab:demografik}
\begin{tabular}{lcc}
\toprule
\textbf{Değişken} & \textbf{Sayı (n)} & \textbf{Yüzde (\%)} \\
\midrule
Cinsiyet & & \\
Kadın & 180 & 72.0 \\
Erkek & 70 & 28.0 \\
\midrule
Yaş Grubu & & \\
20-30 & 100 & 40.0 \\
31-40 & 90 & 36.0 \\
41+ & 60 & 24.0 \\
\bottomrule
\end{tabular}

\scdtablenote{Bu tablo iki sütunlu format içinde gösterilmektedir.}
\end{threeparttable}
\end{table}

% ===========================================================
% ÖRNEK 2: İKİ SÜTUNLU AMA GENİŞ TABLO (table*)
% Açıklama: Tablo tüm sayfa genişliğini kaplar ama yatay değildir.
% Kullanım: Çok sütunlu, geniş tablolar için idealdir.
% NOT: multicols içinde çalışmaz, scdbody dışına taşınmalı!
% ===========================================================
\end{scdbody}

% scdbody dışında - tam genişlik tablo
\begin{table}[H]
\centering
\caption{Engellilik Algısı Alt Boyutları ve Demografik Değişkenlere Göre Karşılaştırma}
\begin{threeparttable}
\label{tab:genis}
\scdtablefont\scdtablesize
\begin{tabular}{lcccccc}
\toprule
\textbf{Alt Boyut} & \textbf{Cinsiyet} & \textbf{n} & \textbf{Ort.} & \textbf{SS} & \textbf{t} & \textbf{p} \\
\midrule
Toplumsal Kabul & Kadın & 180 & 3.82 & 0.76 & 2.45 & 0.015* \\
                & Erkek & 70 & 3.54 & 0.81 & & \\
\midrule
Mesleki Yeterlilik & Kadın & 180 & 4.12 & 0.65 & 1.89 & 0.060 \\
                   & Erkek & 70 & 3.95 & 0.71 & & \\
\midrule
Hak Temelli Yaklaşım & Kadın & 180 & 4.35 & 0.58 & 3.12 & 0.002** \\
                     & Erkek & 70 & 4.02 & 0.69 & & \\
\bottomrule
\end{tabular}

\scdtablenote{*p<0.05, **p<0.01. Bu tablo tam sayfa genişliğinde gösterilmektedir.}
\end{threeparttable}
\end{table}

\begin{scdbody}

\section{TARTIŞMA}

Tablo~\ref{tab:kadin}'te katılımcıların demografik özellikleri sunulmuş, 
Tablo~\ref{tab:genis}'da ise gruplar arası karşılaştırma sonuçlarına yer verilmiştir. 
Bu bölümde bulgular, engellilik olgusunun kuramsal ve uygulamalı boyutları çerçevesinde 
sosyal hizmet perspektifinden tartışılmaktadır.

\subsection{Engellilik Kavramının Sosyal Hizmet Perspektifinden Değerlendirilmesi}

Engellilik, bireyin fiziksel, zihinsel, duyusal veya psikolojik farklılıkları nedeniyle 
toplumsal yaşama tam ve etkin katılımında karşılaştığı engeller bütünü olarak 
tanımlanmaktadır. Sosyal hizmet disiplini, engelliliği yalnızca bireysel bir durum 
olarak değil, aynı zamanda toplumsal ve yapısal faktörlerle şekillenen bir olgu 
olarak ele almaktadır.

Bu yaklaşım, engelliliğin bireyin yetersizliklerinden ziyade çevresel düzenlemeler, 
erişilebilirlik politikaları ve toplumsal tutumlarla doğrudan ilişkili olduğunu 
vurgulamaktadır.

\subsubsection{Biyo-Psiko-Sosyal Yaklaşım ve Güçlendirme}

Sosyal hizmet literatüründe engellilik, biyo-psiko-sosyal model çerçevesinde 
değerlendirilmektedir. Bu model, bireyin fiziksel durumunun yanı sıra psikolojik 
iyi oluşunu ve sosyal çevresiyle kurduğu ilişkileri bütüncül bir biçimde ele almayı 
amaçlamaktadır.

Bu bağlamda sosyal hizmet uygulamaları, engelli bireylerin güçlendirilmesi, 
toplumsal kaynaklara erişimlerinin artırılması ve ayrımcılıkla mücadele edilmesi 
üzerine odaklanmaktadır. Güçlendirme temelli müdahaleler, bireyin kendi yaşamı 
üzerinde söz sahibi olmasını destekleyen temel stratejiler arasında yer almaktadır.

\subsubsection{Toplumsal Dışlanma ve Sosyal Hizmet Müdahaleleri}

Engellilik, uygun destek mekanizmalarının bulunmaması durumunda toplumsal 
dışlanma riskini artırabilmektedir. Sosyal hizmet disiplini, bu dışlanmayı önlemek 
amacıyla savunuculuk, politika geliştirme ve toplumsal farkındalık çalışmaları 
yürütmektedir.

Bu müdahaleler, engelli bireylerin yalnızca hizmet alıcısı değil, aynı zamanda 
hak sahibi ve aktif toplumsal aktörler olarak konumlandırılmasını hedeflemektedir.


% ===========================================================
% ÖRNEK 3: SÜTUN İÇİ YATAY TABLO (ROTATED) - sol sütunda
% Açıklama: Tablo sütun içinde 90 derece döner. 
% Metin (Sonuç kısmı) bu tablonun yanındaki sütundan devam eder.
% Kullanım: Geniş tabloları tek sütuna sığdırmak için.
% ===========================================================

% -- TABLO 1 (Kadın) --
\begin{table}[H]
\centering
\rotatebox{90}{%
  \begin{minipage}{8cm} % Tablo genişliği
    \centering
    \captionof{table}{Kadın Katılımcılar - Korelasyon Matrisi}\label{tab:kadin}
    \begin{threeparttable}
    \scdtablefont\scdtablesize
    \begin{tabular}{lcccc}
      \toprule
      \textbf{Değişken} & \textbf{1} & \textbf{2} & \textbf{3} & \textbf{4} \\
      \midrule
      1. Yaş & 1.00 & & & \\
      2. Deneyim & .75** & 1.00 & & \\
      3. Kabul & -.20* & -.25* & 1.00 & \\
      4. Algı & -.18* & -.22* & .85** & 1.00 \\
      \bottomrule
    \end{tabular}
    \scdtablenote{*p<0.05, **p<0.01.}
    \end{threeparttable}
  \end{minipage}
}
\end{table}
Engellilik, bireyin fiziksel, zihinsel, duyusal veya psikolojik farklılıkları nedeniyle toplumsal yaşama tam ve etkin katılımında karşılaştığı engeller bütünü olarak tanımlanmaktadır. Sosyal hizmet disiplini, engellilik olgusunu biyo-psiko-sosyal bir perspektifle ele alarak, bireylerin güçlendirilmesi ve toplumsal dışlanmanın önlenmesi adına çeşitli müdahale stratejileri geliştirmektedir.

Engellilik, bireyin fiziksel, zihinsel, duyusal veya psikolojik farklılıkları nedeniyle toplumsal yaşama tam ve etkin katılımında karşılaştığı engeller bütünü olarak tanımlanmaktadır. Sosyal hizmet disiplini, engellilik olgusunu biyo-psiko-sosyal bir perspektifle ele alarak, bireylerin güçlendirilmesi ve toplumsal dışlanmanın önlenmesi adına çeşitli müdahale stratejileri geliştirmektedir.

Engellilik, bireyin fiziksel, zihinsel, duyusal veya psikolojik farklılıkları nedeniyle toplumsal yaşama tam ve etkin katılımında karşılaştığı engeller bütünü olarak tanımlanmaktadır. Sosyal hizmet disiplini, engellilik olgusunu biyo-psiko-sosyal bir perspektifle ele alarak, bireylerin güçlendirilmesi ve toplumsal dışlanmanın önlenmesi adına çeşitli müdahale stratejileri geliştirmektedir. % -- TABLO 2 (Erkek) --
\begin{table}[H]
\centering
  \begin{minipage}{9cm} 
    \centering
    \captionof{table}{Erkek Katılımcılar - Korelasyon Matrisi}\label{tab:erkek}
    \begin{threeparttable}
    \scdtablefont\scdtablesize
    \begin{tabular}{lcccc}
      \toprule
      \textbf{Değişken} & \textbf{1} & \textbf{2} & \textbf{3} & \textbf{4} \\
      \midrule
      1. Yaş & 1.00 & & & \\
      2. Deneyim & .80** & 1.00 & & \\
      3. Kabul & -.15 & -.18* & 1.00 & \\
      4. Algı & -.12 & -.16 & .78** & 1.00 \\
      \bottomrule
    \end{tabular}
    \scdtablenote{*p<0.05, **p<0.01.}
    \end{threeparttable}
  \end{minipage}
\end{table}

\section{SONUÇ}
Bu çalışma, engellilik alanında çalışan sosyal hizmet uzmanlarının farkındalık düzeylerinin artırılması gerektiğini göstermektedir. Tablo~\ref{tab:kadin} ve Tablo~\ref{tab:erkek}'da cinsiyete göre korelasyonlar sunulmaktadır.

\begin{scdreferences}
Acak, M., \& Narinç, Ç. (2020). Eğitim fakültesi öğrencilerinin engellilere yönelik tutumlarının ince-lenmesi (Malatya ili örneği). GERMENİCA Beden Eğitimi ve Spor Bilimleri Dergisi, 1(1), 18-27.

Akardere, S. S. (2005). İşverenlerin engelli çalışanlara yönelik tutumları. Yüksek lisans tezi. Marmara Üniversitesi: İstanbul.

Akhidenor, C. D. (2007). Nigerians' attitudes toward people with disabilities. Doctoral dissertation. Capella University: ABD

Aktaş, C., Küçüker, S. (2002). Bilişsel-duyuşsal odaklı bir programın ilköğretim öğrencilerinin fiziksel engelli yaşıtlarına yönelik sosyal kabul düzeylerine etkisinin incelenmesi. Ankara Üniversitesi Eğitim Bilimleri Fakültesi Özel Eğitim Dergisi, 3(2), 15-25.

Alahmari, K. A., Rengaramanujam, K., Reddy, R. S., Silvian Samuel, P., Ahmad, I., Nagaraj Kakaraparthi, V., \& Tedla, J. S. (2021). Effect of disability-specific education on student attitudes toward people with disabilities. Health Education \& Behavior, 48(4), 532-539.

Alyüz, S.B.A. (2022). Engelli ve engelli sorunlarında yeni bir perspektif sosyal dışlanmadan sosyal entegrasyon ve sosyal içermeye yönelim. Nevşehir Hacı Bektaş Veli Üniversitesi SBE Dergisi, 12(1), 457-471.

Apaydın, R., \& Barış, İ. (2021). Toplumda engelli bireylere yönelik tutumun sağlık çalışanları bağlamında değerlendirilmesi. Ufkun Ötesi Bilim Dergisi, 21(1), 22-39.

Aslan, Ö. F., \& Küçükali, A. (Eds.). (2024). Sosyal, iktisadi, beşerî ve yönetim alanları perspektifinde sorunlara bakış (Cilt 3). Efe Akademi Yayınları.

Au, K. W., \& Man, D. W. (2006). Attitudes toward people with disabilities: a comparison between health care professionals and students. International Journal of Rehabilitation Research, 29(2), 155-160.

Babik, I., Gardner, E. S. (2021). Factors affecting the perception of disability: a developmental perspective. Frontiers in Psychology, 12, 702166.

Barr, J. J., \& Bracchitta, K. (2015). Attitudes toward individuals with disabilities: the effects of contact with different disability types. Current Psychology, 34, 223-238.

Baykan, Z., Naçar, M., Şenol, V., \& Çetinkaya, F. (2018). Erciyes Üniversitesi akademik personelinde engellilik farkındalığı. Mersin Üniversitesi Sağlık Bilimleri Dergisi, 11(1), 50-61.

Bennwik, I. H. B., Oterholm, I., \& Kelly, B. (2023). ‘Disability is not a word we use’: social workers' professional judgements about support for disabled young people leaving care. Child \& Family Social Work, 28(2), 443-453.

Bingöl, O. (2015). Genel bağlarıyla sosyal hizmet ve toplum. Mavi Atlas, 5, 69-77.

Bircan, B., Bektaş, G., \& Aytaç, B. (2020). 360 derece performans değerlendirme sistemi ve özel bir sağlık kuruluşu uygulama örneği. Acıbadem Üniversitesi Sağlık Bilimleri Dergisi, 1(1), 567-573.

Borozancı, İ., Sezer, S., Türe., D., Panış, B. İ., Arıcı, A., Bolat, G. B. (2021). Pandemi sürecinde fiziksel engelli bireylerin sorun ve beklentileri üzerine nitel bir inceleme. Ufkun Ötesi Bilim Dergisi, 21(2), 269-289.

Boztepe, Ö., \& Mecek, M. (2023). Belediye çalışanlarının engelli bireylere yönelik hizmet sunumunda bireysel tutumlarının incelenmesi: Afyonkarahisar Belediyesi çalışanları örneği. İktisadi ve İdari Yaklaşımlar Dergisi, 5(2), 123-146

Büyüköztürk, Ş., Kılıç Çakmak, E., Akgün, Ö. E., Karadeniz, Ş., Demirel, F. (2013). Bilimsel Araştırma Yöntemleri. 14. Baskı, Ankara, Pegem Akademi Yayınları.

Corrigan, P. W., Morris, S. B., Michaels, P. J., Rafacz, J. D., \& Rüsch, N. (2012). Challenging the public stigma of mental illness: a meta-analysis of outcome studies. Psychiatric services, 63(10), 963-973.

Çalbayram, N. Ç., Aker, M. N., Akkuş, B., Durmuş, F. K., \& Tutar, S. (2018). Sağlık Bilimleri Fakültesi öğrencilerinin engellilere yönelik tutumları. Ankara Sağlık Bilimleri Dergisi, 7(1), 30-40.

Çalışkan, H., \& Şahin, H. (2024). Üniversite öğrencilerinin engellilere yönelik algılarının ve bilişsel ve duygulanımsal empati düzeylerinin incelenmesi. 1. Third Sector Social Economic Review, 59(3), 1788-1801. 

Çetin, B. I. (2017). Sanayi Devrimi’nden 21. yüzyıla batı dünyasında engellilik. Sosyal Güvenlik Dergisi, 7(1), 91-122.

Çitil, M., \& Üçüncü, M. K. (2018). Türkiye’de engelli hakları ve Engelliler Hukuku’nun durumu. Türkiye Adalet Akademisi Dergisi, (35), 233-278.

Çoşkun., R., Altunışık., R., Yıldırım., E. (2017). Sosyal Bilimlerde Araştırma Yöntemleri SPSS Uygulamalı, (Güncellenmiş 9. Baskı), Sakarya: Sakarya Yayıncılık.

Dayı, E., Açıkgöz, G., Elçi, A.N. (2020). Güzel Sanatlar eğitimi bölümü öğretmen adaylarının engelli öğrencilere yönelik metaforik algıları (Gazi Üniversitesi örneği). Ankara Üniversitesi Eğitim Bilimleri Fakültesi Özel Eğitim Dergisi, 21(1), 95-122.

Demir, C. (2023). Beden eğitimi öğretmenlerinin mesleki benlik saygıları ile engellilere yönelik tutumlarının karşılaştırılması (Bağcılar İlçesi Örneği). Yüksek Lisans Tezi. Gelişim Üniversitesi: İstanbul.

Demirbilek, B., Öcal, Y. K., \& Karabulut, E. O. (2024). Spor bilimleri öğrencilerinin engellilere yönelik algı düzeylerinin incelenmesi. Uluslararası Bozok Spor Bilimleri Dergisi, 5(1), 1-13.

Diğer, H., Yıldız, A. (2021). Engelli bireylere yönelik algı ölçeği: Geçerlik ve güvenirlik çalışması. Fırat Üniversitesi Sosyal Bilimler Dergisi, 31(2), 807-822.

Doğan Özdemir, S. (2021). Özel Eğitim ve Rehabilitasyon Merkezlerinde çalışan eğitimcilerin engellilere yönelik tutumlarının incelenmesi. Yüksek Lisans Tezi. Gelişim Üniversitesi: İstanbul.

Erdoğdu, M. Y. (2008). Duygusal zekanın bazı değişkenler açısından incelenmesi. Elektronik Sosyal Bilimler Dergisi, 7(23), 62-76.

Eroğlu, S., \& Gökkaya, D. (2025). Sağlık Hizmetleri Sistemine Çok Boyutlu Güven ve Sağlık Okuryazarlığı Arasındaki İlişkinin İncelenmesi: Bir Alan Araştırması. Celal Bayar Üniversitesi Sağlık Bilimleri Enstitüsü Dergisi, 12(1), 37-47.

F. Antonak, R., \& Livneh, H. (2000). Measurement of attitudes towards persons with disabilities. Disability and rehabilitation, 22(5), 211-224.

Friedman, C., \& Owen, A. L. (2017). Defining disability: Understandings of and attitudes towards ableism and disability. Disability Studies Quarterly, 37(1).

Furnham, A., \& Pendred, J. (1983). Attitudes towards the mentally and physically disabled. British Journal of Medical Psychology, 56(2), 179-187.

Gedik, Z., \& Toker, H. (2018). Üniversite öğrencilerinin engelli bireylere yönelik tutumları ve sosyal beğenirlik düzeyleri. Yükseköğretim ve Bilim Dergisi, 8(1). 111-116.

Gencer, M. (2019). Engellilik durumu ve engellilere yönelik toplumsal algı: Kütahya örneği. Yüksek Lisans Tezi. Dumlupınar Üniversitesi SBE, Kütahya.

Genç, Y., Çavuşoğlu, O., \& Çöpoğlu, M. (2020). Sosyal politika geliştirme ve sosyal hizmet uygulamalarında yerelleşme: Sakarya Büyükşehir Belediyesi Sosyal Gelişim Merkezi örneği. Sosyal Politika Çalışmaları Dergisi, 20(46), 235-272.

George, D. \& Mallery, M. (2010). SPSS for windows step by step: a simple guide and reference, 17.0 update (10a ed.) Boston: Pearson

Gill, F., Kroese, B. S., \& Rose, J. (2002). General practitioners' attitudes to patients who have learning disabilities. Psychological Medicine, 32(8), 1445-1455.

Girli, A., Sarı, H. Y., Kırkım, G., \& Narin, S. (2016). University students’ attitudes towards disability and their views on discrimination. International Journal of Developmental Disabilities, 62(2), 98-107.

Gülen, B., \& Çakmak, A. (2024). Üniversite öğrencilerinin engellilere yönelik algıları ile sosyal-duygusal yetkinlikleri arasındaki ilişkinin incelenmesi. Premium e-Journal of Social Sciences (PEJOSS), 8(42), 653-663.
\end{scdreferences}

\end{scdbody}

\end{document}
